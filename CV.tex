\documentclass{resume} % include resume.cls file
\usepackage[unicode, pdftex]{hyperref}
\usepackage{xcolor}
\usepackage[left=0.75in,top=0.6in,right=0.75in,bottom=0.6in]{geometry}
\usepackage{enumitem}

\newcommand{\link}[2]{\href{#1}{\textcolor{blue}{\underline{#2}}}}
\newcommand{\subheader}[1]{\textbf{\textit{#1}}}
\newcommand{\timestamp}[1]{\hfill {\small \textbf{#1}}}
\newcommand{\datedsubheader}[2]{\textbf{\textit{#1}} \timestamp{#2}}
\newcommand{\longversion}[1]{
\ifdefined\LONG
	#1
\fi
}
\newcommand{\note}[1]{\textcolor{gray}{#1}}

\setlist[itemize]{nosep,topsep=-8pt}
\name{Artyom Zaporozhets}


\begin{document}
	\textit{Junior backend developer @ Yandex N.V.} \\
	\textit{Student @ MSU Sternberg astronomical institute, Moscow} 

	\begin{rSection}{General}
		\begin{tabular}{@{} >{\bfseries}l @{\hspace{6ex}} l }
			Email & zaporozhetc.aa19@physics.msu.ru \\
			GitHub & \link{https://github.com/Kraysent}{@Kraysent} \\
			Telegram & \link{https://t.me/kraysent}{@kraysent}
		\end{tabular}

		\begin{tabular}{@{} >{\bfseries}l @{\hspace{6ex}} l }
			Born & November 16, 2001 (21 y.o.) \\
			Citizenship & Russia \\
			Place of residence & Russia, Moscow
		\end{tabular}
	\end{rSection}		

	\begin{rSection}{Education}
		\datedsubheader{Lomonosov Moscow State University}{September 2019 - May 2025 (expected)}
		\\ Master of Science
		\\ Physics Faculty, Department of Astronomy
		\\ Sternberg astronomical institute

		\datedsubheader{Traektoria School of Astrophysics}{August 2016 - August 2019}

		\datedsubheader{33 Lyceum, Rostov-on-Don}{September 2008 - June 2019}
		\longversion{
			\\ \note{In-depth study of mathematics \& physics}
		}
	\end{rSection}
	
	\begin{rSection}{Work experience}
		\longversion{
			\note{Developed payment processing system on microservices. Worked with AWS SQS, Docker, Yandex's internal deployment, continuous integration, version control, monitoring and task tracking systems.}
		}

		\datedsubheader{Yandex N.V., FinOps division, billing systems}{since December 2021}
		\begin{itemize}
			\item Junior Python/Go backend developer \timestamp{since June 2022}
			\item Intern Python/Go backend developer \timestamp{December 2021 - June 2022}
		\end{itemize}
    \end{rSection}
	
	\begin{rSection}{Interests and Skills}
		\begin{tabular}{@{} >{\bfseries}l @{\hspace{6ex}} l }
			Spoken languages & fluent in English and Russian, beginner in French \\
			Programming languages & Go, Python (numpy, matplotlib, astropy), C++\\
			Software & LaTeX, Git, GitHub Actions, Docker, Unix, AWS SQS
		\end{tabular}

		\subheader{Language Exam Results}
		\begin{itemize}
			\item First Cambridge English, 165, Grade C \timestamp{2019}
		\end{itemize}

		\vspace{1em}

		\subheader{Online courses}
		\begin{itemize}
			\item Machine learning in applied tasks \textit{from MSU}. \timestamp{September - December 2022}
			\item \link{https://coursera.org/share/d2d0be1f1b9f3e71fc36ca28fb12976f}{Version control with Git} \textit{from Atlassian} \timestamp{May 2020}
			\item The skill of development in modern C++ \textit{from MIPT \& Yandex}. \timestamp{January - March 2020}

			\link{https://coursera.org/share/9ae4ca0b1b59871cd100cd8ffb3d181d}{White belt}, \link{https://coursera.org/share/ef873d3813da5cd7eed359eb3126222e}{Yellow belt}
		\end{itemize}
	\end{rSection}

	\begin{rSection}{Science}
		\longversion{
			\note{
				Worked with different scientific Python packages such as numpy, scipy, matplotlib, pandas and astropy. 
				Have a lot of experience with LaTeX and Unix.
			}
		}

		\subheader{Publication}
		\begin{itemize}
			\item Random Sky Radio Sources, A.A.Zaporozhets, O.V.Verkhodanov \timestamp{2019}
		
			Astrophysical Bulletin, vol. 74, p. 265 (\link{http://www.sao.ru/Doc-k8/Science/Public/Bulletin/Vol74/N3/ASPB265.pdf}{Russian version})
		\end{itemize}

		\vspace{1em}

		\subheader{Programming projects}
		\begin{itemize}
			\item \link{https://github.com/Kraysent/OMTool}{Galaxy evolution modeling tool}, Python \timestamp{since June 2021}
			\item \link{https://github.com/Kraysent/XBodyModel}{Numerical model of N-body task}, Rust \timestamp{2020}
			\item \link{https://github.com/Kraysent/Gravity-Model}{Numerical model of N-body task}, C\# \timestamp{2019}
		\end{itemize}

		\longversion{
			\vspace{1em}

			\subheader{Conferences}
			\begin{itemize}
				\item Physics of Space 2023, Ekaterinburg \timestamp{January - February 2023}
				
				Presentation: \link{https://www.overleaf.com/read/mnwwgvkqxdky}{Russian version}
				\item High Energy Astrophysics today and tomorrow 2022, Moscow \timestamp{December 2022}
				
				Poster: \link{https://www.overleaf.com/read/cwyptqpmdtdf}{Russian version}
			\end{itemize}
		}
	\end{rSection}
\end{document}