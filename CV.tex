\documentclass{resume} % include resume.cls file
\usepackage[unicode, pdftex]{hyperref}
\usepackage{xcolor}
\usepackage[left=0.75in,top=0.6in,right=0.75in,bottom=0.6in]{geometry}
\name{Artyom Zaporozhets}
\newcommand{\link}[2]{\href{#1}{\textcolor{blue}{\underline{#2}}}}

\begin{document}
	\textit{Junior backend developer @ Yandex N.V.} \\
	\textit{Student @ MSU Sternberg astronomical institute, Moscow} 

	\begin{rSection}{Contacts}
		\textbf{Email: } zaporozhetc.aa19@physics.msu.ru
		\\ \textbf{GitHub: } \link{https://github.com/Kraysent}{https://github.com/Kraysent}
	\end{rSection}		

	\begin{rSection}{General}
		\textbf{Date of birth:} November 16, 2001
		\\ \textbf{Citizenship:} Russia
		\\ \textbf{Place of residence:} Russia, Moscow
	\end{rSection}

	\begin{rSection}{Education}
		\textbf{Lomonosov Moscow State University} \hfill \textit{September 2019 - May 2025 (expected)}
		\\ Master of Science
		\\ Physics Faculty, Department of Astronomy

		\textbf{Traektoria School of Astrophysics} \hfill \textit{August 2016 - August 2019}

		\textbf{33 Lyceum, Rostov-on-Don} \hfill \textit{September 2008 - June 2019}
		\\ In-depth study of mathematics \& physics
	\end{rSection}
	
	\begin{rSection}{Work experience}
		$\bullet$ Yandex N.V., FinOps division, billing systems \hfill \textit{since December 2021}
		\begin{itemize}
			\item Junior Python/Go backend developer \hfill \textit{since June 2022}
			\item Intern Python/Go backend developer \hfill \textit{December 2021 - June 2022}
		\end{itemize}
    \end{rSection}
    
    \begin{rSection}{Science}
		\textit{Publication:}
		\\ $\bullet$ Random Sky Radio Sources \link{http://www.sao.ru/Doc-k8/Science/Public/Bulletin/Vol74/N3/ASPB265.pdf}{(Russian)} \hfill \textit{2019}
		\\ A.A.Zaporozhets, O.V.Verkhodanov, Astrophysical Bulletin, vol. 74, p. 265

		\textit{Programming projects:}
		\\ $\bullet$ \link{https://github.com/Kraysent/OMTool}{Galaxy evolution modeling tool}, Python \hfill \textit{since June 2021}
		\\ $\bullet$ \link{https://github.com/Kraysent/XBodyModel}{Numerical model of N-body task}, Rust \hfill \textit{2020}
		\\ $\bullet$ \link{https://github.com/Kraysent/Gravity-Model}{Numerical model of N-body task}, C\# \hfill \textit{2019}
	\end{rSection}
	
	\begin{rSection}{Interests and Skills}
		\begin{tabular}{@{} >{\bfseries}l @{\hspace{6ex}} l }
			Programming & Go, Python (numpy, matplotlib, astropy), C++, C\#\\
			Software & LaTeX, Git, Docker \\
			Languages & English (fluent), Russian (fluent)
		\end{tabular}

		\textit{Language Exam Results:}
		\\ $\bullet$ First Cambridge English, 165, Grade C \hfill \textit{2019}
		
		\textit{Online courses:}
		\\ $\bullet$ Machine learning in applied tasks \textit{from MSU}. \hfill \textit{September - December 2022}
		\\ $\bullet$ \link{https://coursera.org/share/d2d0be1f1b9f3e71fc36ca28fb12976f}{Version control with Git} \textit{from Atlassian} \hfill \textit{May 2020}
		\\ $\bullet$ The skill of development in modern C++ \textit{from MIPT \& Yandex}. \hfill \textit{January - March 2020}
		\\ \link{https://coursera.org/share/9ae4ca0b1b59871cd100cd8ffb3d181d}{White belt}, \link{https://coursera.org/share/ef873d3813da5cd7eed359eb3126222e}{Yellow belt} 
	\end{rSection}
\end{document}