\documentclass{resume} % include resume.cls file
\usepackage[unicode, pdftex]{hyperref}
\usepackage{xcolor}
\usepackage[left=0.4in,top=0.4in,right=0.4in,bottom=0.3in]{geometry}
\usepackage{enumitem}
\usepackage{courier}

\newcommand{\link}[2]{\href{#1}{\textcolor{blue}{\underline{#2}}}}
\newcommand{\subheader}[2]{
	{\textbf{#1}} \ifthenelse{\equal{#2}{}}{}{\timestamp{#2}}
}

\renewcommand{\labelitemi}{$\circ$}

\setlist[itemize]{nosep,topsep=-8pt}
\name{Artyom Zaporozhets}

\begin{document}
	\textit{Researcher @ Special Astrophysical Observatory} \\
	\textit{Student @ Sternberg Astronomical Institute} \\
	\textit{Backend Team Lead @ Yandex} \\

	\begin{rSection}{General}
		\begin{tabular}{@{} >{\bfseries}l @{\hspace{6ex}} l }
			Email & kraysent@gmail.com \\
			LinkedIn & \link{https://www.linkedin.com/in/kraysent/}{https://www.linkedin.com/in/kraysent/} \\
			GitHub & \link{https://github.com/Kraysent}{https://github.com/Kraysent}
		\end{tabular}

		\begin{tabular}{@{} >{\bfseries}l @{\hspace{6ex}} l }
			Born & November 16, 2001 (23 y.o.) \\
			Location & Russian Federation \\
		\end{tabular}
	\end{rSection}

	\begin{rSection}{Education}
		\begin{rSubsection}{Lomonosov Moscow State University}{September 2019 --- May 2025}{}{}{msu}
			\item \subheader{Master of Science}{}
			\begin{itemize}
				\item Physics Faculty, Astronomy Dept, Sternberg astronomical institute
			\end{itemize}
		\end{rSubsection}

		\rSubsectionHeader{traektoria}{Traektoria foundation, Astrophysics school}{August 2016 --- August 2019}
	\end{rSection}

	\begin{rSection}{Research experience}
		\begin{rSubsection}{HyperLEDA database of extragalactic objects}{since January 2024}{}{}{}
			\item Developed a backend for the database of extragalactic objects, including the database itself (Postgres) and a Python web server that allows access to this database. Implemented algorithms to modify this data, from the step of uploading the table by a database administrator to the calculation of physical parameters to user access to this data.
			\item \subheader{Conferences}{}
			\begin{itemize}
				\item National astronomical conference 2024 \timestamp{August 2024}
			\end{itemize}
			\item \textbf{Programming project}: \link{https://github.com/HyperLEDA/db-app}{Backend for the database of extragalactic objects}, Python \timestamp{2024}
		\end{rSubsection}
		\begin{rSubsection}{Wandering black holes in Milky Way}{2022 --- January 2024}{}{}{}
			\item Modelled the path of a black hole in the collision of two galaxies. Implemented software that allowed parallel computation of such models. Predicted its final position today, giving a range of possible coordinates and velocity vectors inside the Galaxy.
			\item \subheader{Conferences}{}
			\begin{itemize}
				\item Physics of Space 2023, Ekaterinburg \timestamp{January --- February 2023}
				
				Presentation: \link{https://www.overleaf.com/read/mnwwgvkqxdky}{Russian version}
				\item High Energy Astrophysics today and tomorrow 2022, Moscow \timestamp{December 2022}
				
				Poster: \link{https://www.overleaf.com/read/cwyptqpmdtdf}{Russian version}
			\end{itemize}

			\item \textbf{Programming project}: \link{https://github.com/Kraysent/OMTool}{Galaxy evolution modeling tool}, Python \timestamp{2022 --- 2023}
		\end{rSubsection}

		\begin{rSubsection}{Computation of the cloudiness of the sky using infrared camera}{2021}{}{}{}
			\item Implemented an algorithm to calculate the percentage of cloudiness and fogginess of the sky. The algorithm used pictures from infra-red camera in Caucasian Mountain Observatory to calculate these values in real time.
			\item \textbf{Programming project}: \link{https://github.com/Kraysent/cloudiness}{Cloudiness algorithm}, Python \timestamp{July 2021}
		\end{rSubsection}

		\begin{rSubsection}{Random Sky Radio Sources}{2019}{}{}{}
			\item Worked on a statistical analysis of radio source data from all-sky surveys. Analysed such surveys to produce a list of 120 possible clusters, 46 of which could have redshifts greater than $0.7$.
			\item \subheader{Scientific publication}{}
			\begin{itemize}
				\item Random Sky Radio Sources, A.A.Zaporozhets, O.V.Verkhodanov \timestamp{2019}

				Astrophysical Bulletin, vol. 74, p. 265 (\link{http://www.sao.ru/Doc-k8/Science/Public/Bulletin/Vol74/N3/ASPB265.pdf}{Version in Russian language})
			\end{itemize}	
		\end{rSubsection}
	\end{rSection}

	\begin{rSection}{Work experience}
		\begin{rSubsection}{Special Astrophysical Observatory of the Russian Academy of Science}{since May 2024}{}{}{sao}
			\item \subheader{Part-time researcher}{since May 2024}
			\begin{itemize}
				\item Worked on the HyperLEDA extragalactic database.
				\item Participated in the Academy of Science grant.
			\end{itemize}
		\end{rSubsection}

		\begin{rSubsection}{Yandex, FinTech division}{since December 2021}{}{}{yandex}
			\item \subheader{Backend team lead}{since September 2024}
			\begin{itemize}
				\item Managed up to 4 people. Developed a backend of a loyalty project including selection of cashback categories, personalization of these categories, a number of product requirements and features that span over multiple teams.
			\end{itemize}
			\item \subheader{Middle Go/Python backend developer}{May 2023 - September 2024}
			\begin{itemize}
				\item Backend feature lead for Yandex Pay loyalty project. Launched a backend that allowed users to choose cashback offers for each month with 100s of requests per second. Managed two people as a part of the project. Communicated with product managers, frontend and backend developers from adjacent services.
				\item Stabilized Yandex Pay accrual platform to allow faster and easier implementation of product ideas for cashback loyalty programs. Implemented regular payments data reconciliation. Mentored an intern.
				\item Implemented asynchronous export of receipt data from Oracle monolith database to PostgreSQL database inside the microservice with Kafka queue as intermediate that allowed for more fault tolerant access to and export of this data.
			\end{itemize}
			\item \subheader{Junior Go/Python backend developer}{June 2022 --- May 2023}
			\begin{itemize}
				\item Implemented service for the automated configuration of 10 microservices that allowed for much faster ``product idea'' --- ``change in configuration'' cycle.
				\item Initialized the infrastructure (balancers, DBs, Kafka queues, etc.) for 10 SOX-compliant microservices in Yandex Cloud to create a new environment for the clients of the platform to integrate to.
			\end{itemize}
			\item \subheader{Intern Go/Python backend developer}{December 2021 --- June 2022}
			\begin{itemize}
				\item Added asynchronous SQS tasks support to a microservice that allowed it to interact with internal services such as task tracker and S3 storage.
				\item Implemented templating engine for dashboards in Grafana to allow standardized creation of similar runtime monitoring dashboards for different services.
			\end{itemize}
		\end{rSubsection}

		\vspace{1em}

		\begin{rSubsection}{Traektoria foundation (non-profit), Astrophysics school}{August 2020 --- August 2023}{}{}{traektoria}
			\item \subheader{Tutor}{}
			\begin{itemize}
				\item Worked with 14--18 year old students learning math, physics, astronomy and programming. Assisted teachers with lectures. Helped students with study material and communication with each other and teachers.
			\end{itemize}
		\end{rSubsection}
    \end{rSection}

	\begin{rSection}{Interests and Skills}
		\begin{tabular}{@{} >{\bfseries}l @{\hspace{6ex}} l }
			Spoken languages & fluent in English and Russian, beginner in French \\
			Devtools \& Software & Git, GitHub, Docker, Kafka, SQS, Grafana, Terraform, Yandex Cloud\\
			Programming languages & Go, Python, Bash, some Rust and C++\\
		\end{tabular}

		\subheader{Language Exams}{}
		\begin{itemize}
			\item FCE, B2 \timestamp{2019}
			\item IELTS 8.0 \timestamp{2024}
		\end{itemize}

		\vspace{1em}

		\subheader{Online courses}{}
		\begin{itemize}
			\item Machine learning in applied tasks \textit{from MSU}.\ \timestamp{September --- December 2022}
			\item \link{https://coursera.org/share/d2d0be1f1b9f3e71fc36ca28fb12976f}{Version control with Git} \textit{from Atlassian} \timestamp{May 2020}
			\item The skill of development in modern C++ \textit{from MIPT \& Yandex}.\ \timestamp{January --- March 2020}

			\link{https://coursera.org/share/9ae4ca0b1b59871cd100cd8ffb3d181d}{White belt}, \link{https://coursera.org/share/ef873d3813da5cd7eed359eb3126222e}{Yellow belt}
		\end{itemize}
	\end{rSection}
\end{document}